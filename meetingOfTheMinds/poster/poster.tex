\documentclass[landscape, 20pt, blockverticalspace=0.5cm]{tikzposter}
\usetikzlibrary{arrows.meta}


%------------------------------------------
% Fonts
%------------------------------------------
\usepackage[T1]{fontenc}
%\usepackage{lmodern} removed lmodern b/c itmesses up sums
\usepackage{textcomp}
%\renewcommand*{\sfdefault}{lmss} % other sf fonts: cmss, lmss, pag, phv
\renewcommand{\familydefault}{\sfdefault}
\usepackage{xcolor}
\usepackage{array}
\usepackage{graphicx}
\usepackage{amsmath,amssymb,amsthm}
%------------------------------------------
% Poster options & visual elements
%------------------------------------------
\geometry{paperwidth=30in,paperheight=42in}
\usetheme{Autumn}
\usecolorstyle{Denmark}
\useblockstyle{Default}
\useinnerblockstyle{Default}
\setlength{\textwidth}{42in}
\let\oldblock\block
\renewcommand{\block}[2]{\oldblock{\huge #1}{\Large #2}}

% colors

\definecolor{darkred}{HTML}{7A0000}

\colorlet{titlebgcolor}{darkred}

\colorlet{blocktitlefgcolor}{darkred}
\colorlet{innerblocktitlebgcolor}{darkred}



% theorem boxes, etc.
\newtheorem*{thm}{Theorem}

\theoremstyle{definition}
\newtheorem*{defn}{Definition}


%------------------------------------------
% Math-related packages and commands
%------------------------------------------
\newcommand{\A}{\mathcal{A}}
\newcommand{\G}{\mathcal{G}}
\renewcommand{\P}{\mathfrak{A}}
\newcommand{\C}{\mathfrak{C}(\Am,\e)}
\newcommand{\Z}{\mathbb{Z}}
\newcommand{\Q}{\mathbb{Q}}
\newcommand{\2}{\textbf{2}}
\newcommand{\Am}{\textbf{A}}
\newcommand{\del}{\partial}
\newcommand{\vv}{\bar{v}}
\renewcommand{\r}{\bar{r}}
\newcommand{\e}{\bar{e}}
\newcommand{\T}{\mathcal{T}}

%------------------------------------------
% Title information
%------------------------------------------
\title{Extensions of Abelian Automaton Groups}

\author{Chris Grossack\\ (Advisor: Klaus Sutner)}

%------------------------------------------
% Begin Document
%------------------------------------------
\begin{document}
\maketitle[titlegraphictotitledistance=-1cm] % rescale title in class file b/c option for it doesn't work
\hspace{2in}
\vspace*{5in}
\begin{columns}
%------------------------------------------
% First column
%------------------------------------------
    \column{0.28}
    
    \block{Background}
    {
      For our purposes, a \textbf{Mealy Automaton} is a tuple $\A = (S, \tau)$
      where $S$ is the \textbf{State Set}, and $\tau : S \times \2 \to S \times \2$ 
      is the \textbf{transition function}. 
      Given a state $s \in S$, we can treat it as a length preserving function 
      $\underline{s} : \2^* \to \2^*$ as follows:
      \begin{align*}
        \underline{s}(\varepsilon) &= \varepsilon\\
        \underline{s}(ax)       &= a' \underline{s'}(x) 
        ~~~(\text{where } (s', a') = \tau(s,a))
      \end{align*}

      For us, each $\tau(s,-) : \2 \to \2$ will be a permutation,
      as this will ensure inverses exist when we define $\G(\A)$,
      the group of functions the automaton generates.

      A state is called \textbf{Odd} if its permutation flips its
      input, and \textbf{Even} if it copies its input.
    }

    \block{Groups}
    {
      We define $\G(\A)$ to be the group generated by 
      $\{ \underline{s} | s \in \A \}$. We will restrict
      ourselves to Abelian groups, and write our groups 
      additively.

      For $f \in \G(\A)$, put $\del_0 f$ as the unique function
      such that for all $w \in \2^*$, $f(0w) = (f(0))(\del_0 f)(w)$.
      Define $\del_1 f$ symmetrically.

      In the abelian case, $\G(\A)$ will always be $\Z^m$ or $( \Z/2 \Z )^m$.
      We restrict our attention to the $\Z^m$ case.
    }

    \block{An Automaton: $\A^3_2$}
    {
      Taking $\Am = \begin{pmatrix} -1 & 1 \\ -\frac{1}{2} & 0 \end{pmatrix}$,
      and $\e = \begin{pmatrix} 3 \\ 2 \end{pmatrix}$ gives the following
      machine, called $\A^3_2$. Here $f = \begin{pmatrix} 1 \\ 0 \end{pmatrix}$,
      $f_0 = \begin{pmatrix} 0 \\ 1 \end{pmatrix}$, and 
      $f_1 = \begin{pmatrix} -2 \\ -2 \end{pmatrix}$:

      \bigskip

      \begin{center}
      \begin{tikzpicture}[scale=0.5]
      \tikzstyle{every node}+=[inner sep=0pt]
      \draw [black] (26.6,-12.1) circle (3);
      \draw (26.6,-12.1) node {\huge $f$};
      \draw [black] (17.4,-25.6) circle (3);
      \draw (17.4,-25.6) node {\huge $f_1$};
      \draw [black] (35.7,-25) circle (3);
      \draw (35.7,-25) node {\huge $f_0$};
      \draw [black] (16.648,-22.708) arc (-174.31762:-254.2298:9.658);
      \fill [black] (16.65,-22.71) -- (17.07,-21.86) -- (16.07,-21.96);
      \draw (17.67,-14.97) node [left] {\huge $1/0$};
      \draw [black] (33.761,-27.277) arc (-48.14861:-128.09563:11.117);
      \fill [black] (33.76,-27.28) -- (32.83,-27.44) -- (33.5,-28.18);
      \draw [black] (29.501,-12.822) arc (67.8011:2.5992:10.451);
      \fill [black] (29.5,-12.82) -- (30.05,-13.59) -- (30.43,-12.66);
      \draw [black] (32.758,-24.474) arc (-108.88288:-180.71682:9.832);
      \fill [black] (32.76,-24.47) -- (32.16,-23.74) -- (31.84,-24.69);
      \draw (27.31,-22.21) node [left] {\huge $0/1$};
      \end{tikzpicture}
      \end{center}
      \[ 
        f(0110) = 1 f_0(110) = 11 f(10) = 110 f_1(0) = 1100
      \]

      \[
        \del_0 f = f_0 \text{ and } \del_0 f_0 = \del_1 f_0 = f
      \]
    }


%------------------------------------------
% Second column
%------------------------------------------
    \column{0.35}
    
    
    \block{Main Theorems}
    {
      \begin{thm}
        Every nontrivial abelian automaton $\A$ can be 
        located at $\e_1$ in some $\C$
      \end{thm}

      \begin{thm}
        If $rp = q$ in $\Z[x]$, then 
        $\mathfrak{C}(\Am,p \cdot \e_1) \hookrightarrow \mathfrak{C}(\Am,q \cdot \e_1)$, 
        with a canonical injection $\varphi_r : \vv \mapsto r \cdot \vv$. 
        In particular, if $r$ is a unit, then $p \cdot \G \cong q \cdot \G$.
        This map preserves both the group and residuation structure.
      \end{thm}

      These allow us to completely understand the residuation vector $\e$.
      \begin{itemize}
        \item First find $\r$ such that $\A$ has a state at $\e_1$.
        \item $\A$ is a subautomaton of $\C$ if and only if $p_{\r}$
          divides $p_{\e}$
        \item Also, if $\A$ \emph{is} a subautomaton, then 
          $q p_{\r} = p_{\e}$, and $\A$ is located at $q \cdot \e_1$
      \end{itemize}
    }
    
    \block{Example}
    {
      So $p \cdot \vv \in \mathfrak{C}(\Am, p \cdot \e_1)$, 
      computes exactly the same function as $\vv \in \C$.
      However, most vectors cannot be written as $p \cdot \vv$. 
      What do they do as functions?
      We call such vectors (and their corresponding functions)
      \textbf{Fractional}, due to the following observation and theorem:

      Let $\delta$ be the function computed by $\e_1 \in \C$. 
      Then $3\e_1 \in \mathfrak{C}(\Am, 3 \cdot \e_1)$ should compute the
      same function. So we should expect the function 
      $\e_1 \in \mathfrak{C}(\Am, 3 \cdot \e_1)$ to behave like 
      ``$\frac{1}{3}\delta$'', and in fact it does.
    }

    \block{Initial Object: The Principal Automaton}
    {
      So $\mathfrak{C}(\Am,\e_1)$ is a subautomaton of every $\mathfrak{C}(\Am,p \cdot \e)$,
      and it is reasonable to ask if there is a corresponding finite 
      automaton whose group embeds into the group generated by every other
      automaton with the same matrix.

      Such a machine was found by Okano, and is called the 
      \textbf{Principal Automaton} of a matrix. It is located at $\e_1$
      in $\mathfrak{C}(\Am,\e_1)$ for any matrix $\Am$, and can be
      directly constructed by taking differences of the states in 
      a given automaton $\A$. 
      
      Below is the result of this construction for $\A^3_2$:

\begin{center}
\begin{tikzpicture}[scale=0.5]
\tikzstyle{every node}+=[inner sep=0pt]
\draw [black] (36.4,-15.9) circle (3);
\draw (36.4,-15.9) node {$I$};
\draw [black] (37.8,-18.6) arc (60.34019:-227.65981:2.25);
\fill [black] (35.17,-18.7) -- (34.34,-19.15) -- (35.21,-19.64);
\draw [black] (55.3,-21) circle (3);
\draw (55.3,-21) node {$f_1-f$};
\draw [black] (64,-36.8) circle (3);
\draw (64,-36.8) node {$f_0-f_1$};
\draw [black] (46.9,-36.8) circle (3);
\draw (46.9,-36.8) node {$f-f_0$};
\draw [black] (28,-36.8) circle (3);
\draw (28,-36.8) node {$f_0-f$};
\draw [black] (11.8,-36.8) circle (3);
\draw (11.8,-36.8) node {$f_1-f_0$};
\draw [black] (19.6,-21) circle (3);
\draw (19.6,-21) node {$f-f_1$};
\draw [black] (22.47,-20.13) -- (33.53,-16.77);
\fill [black] (33.53,-16.77) -- (32.62,-16.53) -- (32.91,-17.48);
\draw (29.5,-19.03) node [below] {$0/1$};
\draw [black] (18.27,-23.69) -- (13.13,-34.11);
\fill [black] (13.13,-34.11) -- (13.93,-33.61) -- (13.03,-33.17);
\draw (15,-27.81) node [left] {$1/0$};
\draw [black] (14.8,-36.8) -- (25,-36.8);
\fill [black] (25,-36.8) -- (24.2,-36.3) -- (24.2,-37.3);
\draw [black] (26.59,-34.15) -- (21.01,-23.65);
\fill [black] (21.01,-23.65) -- (20.94,-24.59) -- (21.83,-24.12);
\draw (24.48,-27.74) node [right] {$0/1$};
\draw [black] (30.345,-34.939) arc (122.03013:57.96987:13.397);
\fill [black] (44.56,-34.94) -- (44.14,-34.09) -- (43.61,-34.94);
\draw (37.45,-32.4) node [above] {$1/0$};
\draw [black] (52.4,-20.22) -- (39.3,-16.68);
\fill [black] (39.3,-16.68) -- (39.94,-17.37) -- (40.2,-16.41);
\draw (47.22,-17.84) node [above] {$1/0$};
\draw [black] (56.75,-23.63) -- (62.55,-34.17);
\fill [black] (62.55,-34.17) -- (62.61,-33.23) -- (61.73,-33.71);
\draw (58.98,-30.09) node [left] {$0/1$};
\draw [black] (61,-36.8) -- (49.9,-36.8);
\fill [black] (49.9,-36.8) -- (50.7,-37.3) -- (50.7,-36.3);
\draw [black] (48.31,-34.15) -- (53.89,-23.65);
\fill [black] (53.89,-23.65) -- (53.07,-24.12) -- (53.96,-24.59);
\draw (51.78,-30.06) node [right] {$1/0$};
\draw [black] (44.711,-38.841) arc (-53.966:-126.034:12.343);
\fill [black] (30.19,-38.84) -- (30.54,-39.72) -- (31.13,-38.91);
\draw (37.45,-41.7) node [below] {$0/1$};
\end{tikzpicture}
\end{center}
    }

%----------------------------------------------
% Third column
%----------------------------------------------
 
    \column{0.28}
    
    \block{Complete Automaton}
    {
      \begin{thm}[Nekrashevych and Sidki]
        If $\G$ is an automaton group and $\varphi : \G \to \Z^m$ is a group 
        isomorphism, then there is a matrix $\Am$ of $\Q$-irreducible character
        and an odd vector $\e$ such that if $\Z^m$ is equipped with the following 
        residuation structure, then $\varphi$ preserves residuation:

        If $\vv$ is even:
        \[ \del_0 \vv = \del_1 \vv = \Am \vv \]

        If $\vv$ is odd:
        \[ \del_0 \vv = \Am (\vv - \e) \]
        \[ \del_1 \vv = \Am (\vv + \e) \]

        Further, $\Am$ is unique up to conjugation, and can always be taken to be 
        ``$\frac{1}{2}-integral$'', meaning $\Am$ is of the form 

        \[
        \begin{pmatrix}
          \frac{a_{11}}{2} & a_{12} & \dots  & a_{1n}\\
          \vdots           & \vdots & \ddots & \vdots\\
          \frac{a_{n1}}{2} & a_{n2} & \dots  & a_{nn}\\
        \end{pmatrix}
        \]
        where each $a_{ij} \in \Z$.

        \bigskip

        These matrices all have characteristic polynomial\\
        $\chi = x^n + \frac{1}{2}g(x)$, where $g \in \Z[x]$ and has constant term 
        $\pm 1$.
      \end{thm}
            
      \begin{defn}
        This construction is called $\C$
      \end{defn}

      \begin{thm}
        Every abelian automaton $\A$ can be embedded in $\C$ for some
        $\Am$ and $\e$.
      \end{thm}

      It is natural to wonder what vectors $\e$ admit a given 
      automaton $\A$ as a subautomaton of $\C$. We answer this question here.
    }

    \block{$\Z[x]$-modules}
    {
      $\G(\A)$ naturally allows scalars in $\Z$ by putting 
      \[ n \cdot f = \underbrace{f + f + \ldots + f}_\text{$n$ times} \]
      We allow scalars in $\Z[x]$, polynomials with integer coefficients,
      by embedding $\G(\A)$ in $\C$, and putting 
      $p \cdot \vv = p(\Am^{-1}) \vv$.

      \begin{thm}
        Because $\Am$ has irreducible character, every vector $\vv$ is 
        equal to $p_{\vv} \cdot \e_1$ for some $p_{\vv}$. Here $\e_1$ is
        the unit vector $(1,0,\ldots,0)$.
      \end{thm}
    }

    \block{Acknowledgements}
    {
      This research would not exist without the advice of my advisor 
      Dr Klaus Sutner. There aren't enough thanks for the hours of 
      conversation I enjoyed. 
    }
    
\end{columns}
       
\end{document}
