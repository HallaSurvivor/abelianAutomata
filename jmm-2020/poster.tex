\documentclass[24pt]{tikzposter}

\geometry{paperwidth=48in, paperheight=36in}
\makeatletter
\setlength{\TP@visibletextwidth}{\textwidth-2\TP@innermargin}
\setlength{\TP@visibletextheight}{\textheight-2\TP@innermargin}
\makeatother

%{{{ Preamble

%%%%%%%%{{{ Packages
%%%%%%%%%%%%%%%%%%%%

% we aren't animals - we might use more than ASCII
\usepackage[utf8]{inputenc}
\usepackage[T1]{fontenc}

% obligatory math environments, symbols, and theorems
\usepackage{amsmath, amssymb, amsthm}

% moar symbols
\usepackage{stmaryrd}

% obligatory citation library
\usepackage{natbib}

% sometimes you gotta draw stuff, like _c_ommutative _d_iagrams.
\usepackage{tikz, tikz-cd}

% sometimes you gotta put in pretty pictures
\usepackage{graphicx}

% sometimes you gotta write code
\usepackage{listings}

% proof trees are useful
\usepackage{proof}

% use \mathbbm for bb-numerals, use \bm for bold math symbols.
\usepackage{bbm, bm}

% I like clickable links in pdfs
\usepackage{hyperref}

% Convenient todo-notes and missing figure boxes
\usepackage{todonotes}

% remove paragraph indentation
\usepackage{parskip}

%}}}

%%%%%%%%{{{ Formatting
%%%%%%%%%%%%%%%%%%%%%%

% prevent orphans/widows
\clubpenalty = 10000
\widowpenalty = 10000

% never break words across lines -- hyphens are stupid
\hyphenpenalty 10000
\exhyphenpenalty 10000

%}}}

%%%%%%%%{{{ Environments
%%%%%%%%%%%%%%%%%%%%%%%%

\newtheorem{thm}{Theorem}

\theoremstyle{definition}
\newtheorem{defn}{Definition}

%}}}

%%{{{ Aliases and Commands
%%%%%%%%%%%%%%%%%%%%%%%%%%

%{{{ blackboard letters
\newcommand{\N}{\mathbb{N}}
\newcommand{\Z}{\mathbb{Z}}
\newcommand{\Q}{\mathbb{Q}}
\newcommand{\R}{\mathbb{R}}
\newcommand{\C}{\mathbb{C}}
%}}}

%{{{ categories
\newcommand*{\catFont}[1]{\mathsf{#1}} 
\newcommand*{\catVarFont}[1]{\mathcal{#1}}

\newcommand{\Set}{\catFont{Set}}
\newcommand{\Top}{\catFont{Top}}
\newcommand{\Cat}{\catFont{Cat}}
\newcommand{\Grp}{\catFont{Grp}}
\newcommand{\Mod}{\catFont{Mod}}
\newcommand{\Sub}{\catFont{Sub}}
\newcommand{\FP}{\catFont{FP}}
\newcommand{\Pos}{\catFont{Pos}}
\newcommand{\FinSet}{\catFont{FinSet}}

\newcommand{\cata}{\catVarFont{A}}
\newcommand{\catb}{\catVarFont{B}}
\newcommand{\catc}{\catVarFont{C}}
\newcommand{\catd}{\catVarFont{D}}
\newcommand{\catx}{\catVarFont{X}}
\newcommand{\caty}{\catVarFont{Y}}
\newcommand{\catZ}{\catVarFont{Z}}
%}}}

%{{{ "operators" (words in math mode)
\DeclareMathOperator{\Hom}{Hom}
\DeclareMathOperator{\End}{End}
\DeclareMathOperator{\Aut}{Aut}
\DeclareMathOperator{\im}{im}
\DeclareMathOperator{\coker}{coker}
%}}}

%{{{ arrows
\newcommand{\hookr}{\hookrightarrow}
\newcommand{\hookl}{\hookleftarrow}
\newcommand{\monor}{\rightarrowtail}
\newcommand{\monol}{\leftarrowtail}
\newcommand{\epir}{\twoheadrightarrow}
\newcommand{\epil}{\twoheadleftarrow}
\newcommand{\regepir}{\rightarrowtriangle}
\newcommand{\regepil}{\leftarrowtriangle}
%}}}

%{{{ categorical limits
\newcommand{\limr}{\varinjlim}
\newcommand{\liml}{\varprojlim}
%}}}

%{{{ analysis
\DeclareMathOperator{\dif}{d \!}
\DeclareMathOperator{\Dif}{D \!}
\newcommand{\del}{\partial}
\newcommand*{\abs}[1]{\left \lvert #1 \right \rvert}
\newcommand*{\norm}[1]{\left \lVert #1 \right \rVert}
\newcommand*{\eval}[1]{\left . #1 \right \rvert}

% partial derivative command (taken from commath package)
% usage: \pd[n]{f}{x}
\newcommand*{\pd}[3][]{\ensuremath{
\ifinner
\tfrac{\partial{^{#1}}#2}{ \partial{#3^{#1}} }
\else
\dfrac{\partial{^{#1}}#2}{ \partial{#3^{#1}} }
\fi
}}

% ordinary derivative command (taken from commath package)
% usage: \od[n]{f}{x}
\newcommand*{\od}[3][]{\ensuremath{
\ifinner
\tfrac{\dif{^{#1}}#2}{ \dif{#3^{#1}} }
\else
\dfrac{\dif{^{#1}}#2}{ \dif{#3^{#1}} }
\fi
}}

%}}}

%{{{ algebra
\newcommand{\meet}{\wedge}
\newcommand{\join}{\vee}
\newcommand{\id}{\mathrm{id}}
%}}}

%{{{ topology
\newcommand*{\interior}[1]{ {\kern0pt#1}^{\mathrm{o}} }
%}}}

%{{{ logic
\renewcommand{\diamond}{\lozenge}
\newcommand*{\denote}[1]{\llbracket #1 \rrbracket}
%}}}

%{{{ misc symbols
\newcommand{\teq}{\triangleq}
\newcommand{\fin}{ \subseteq_{\text{fin}} }
\newcommand{\lleft}{\llbracket}
\newcommand{\rright}{\rrbracket}

% define a "danger" symbol for use when something surprising might occur
% https://tex.stackexchange.com/questions/159669/
% 	how-to-print-a-warning-sign-triangle-with-exclamation-point
% use outside of math mode!

\newcommand*{\TakeFourierOrnament}[1]{}}

%}}}

%% project specific aliases (if they exist)
\IfFileExists{../preamble.tex}{\input{../preamble.tex}}{}

%}}}

%% Heading
\author{Chris Grossack}
\title{Extensions of Abelian Automata Groups}
\author{Chris Grossack (advised by Klaus Sutner)}
\institute{Carnegie Mellon University}

\begin{document}
\maketitle

\begin{columns}
%% The left side {{{
  \column{0.25}
  \block{Mealy Automata}
  {
    A \textbf{Mealy Automaton} $\mathcal{A}$ is a finite state 
    machine which encodes a family of continuous functions from 
    Cantor Space to itself. For us, these continuous functions 
    will always be homeomorphisms, and thus we may associate
    to a machine $\mathcal{A}$ a subgroup $\mathcal{G}(\mathcal{A})$
    of the automorphisms of Cantor Space. These are known as 
    \textbf{Automata Groups} in the literature.

    \bigskip

    Classifying all groups generated by even 3-state machines is still
    and open problem, so we will focus attention on those which generate
    abelain groups.

    \bigskip

    We can define two operations from $\mathcal{G}(\mathcal{A})$ to 
    itself called \textbf{Residuation}, where the $0$-residual
    of $f$ is defined to be the unique function $\del_0 f$ so that 
    \[ f(0s) = f(0) (\del_0 f)(s) .\]

    The $1$-residual $\del_1 f$ is defined analogously.

    \bigskip

    A function $f$ is called \textbf{even} if it copies its first input bit,
    that is $f(as) = a \del_a f(s)$, and is called \textbf{odd} otherwise.

    \bigskip

    It is a theorem of Sutner that, in the abelian case, $f$ is even 
    if and only if $\del_0 f = \del_1 f$.
  }

  \block{Past Results}
  {
    In their paper ``Automorphisms of the binary tree: 
    State-closed subgroups and dynamics of 1/2-endomorphisms'', 
    Nerkashevych and Sidki show that abelian automata groups are
    isomorphic to integer lattices, and moreover, there is a 
    ``1/2-integral'' matrix $\mathbf{A}_\mathcal{A}$ of irreducible character 
    so that residuation lifts to an affine map. Succinctly, for some $\varphi$:


    \[ \varphi : \mathcal{G}(\mathcal{A}) \cong \Z^m \]

    \[ 
      \varphi(\del_0 f) = 
      \begin{cases} 
        \mathbf{A} \varphi(f) & \text{$f$ even} \\
        \mathbf{A} (\varphi(f) - \overline{e}) & \text{$f$ odd}
      \end{cases}
    \]

    \bigskip

    \[ 
      \varphi(\del_1 f) = 
      \begin{cases} 
        \mathbf{A} \varphi(f) & \text{$f$ even} \\
        \mathbf{A} (\varphi(f) + \overline{e}) & \text{$f$ odd}
      \end{cases}
    \]

    \bigskip
    \bigskip

    Moreover, $\varphi$ can be chosen so that the first component of 
    $\varphi(f)$ is even iff $f$ is even. Under this additional constraint,
    $\overline{e}$ must be odd, and we can put $\mathbf{A}$ into 
    rational canonical form $(a_{i} \in \Z)$:

    \[
      \begin{pmatrix}
        \frac{a_1}{2}     & 1      & 0      & \dots  & 0\\
        \frac{a_2}{2}     & 0      & 1      & \dots  & 0\\ 
        \vdots            & \vdots & \vdots & \ddots & \vdots\\
        \frac{a_{n-1}}{2} & 0      & 0      & \dots  & 1\\
        \frac{a_n}{2}     & 0      & 0      & \dots  & 0\\
      \end{pmatrix}
    \]

    \[ \chi_\mathbf{A} \text{is $\Q$-irreducible} \]

    Finally, our machines are finite if and only if the spectral radius of
    $\mathbf{A}$ is $\rho(\mathbf{A}) < 1$.
  }


%%}}}

%% The center {{{
  \column{0.5}
  \block{The Questions}
  {
    By choosing a 1/2-integral matrix $\mathbf{A}$ and an odd residuation
    vector $\overline{e}$, we can view $\Z^m$ as a Mealy Automaton with
    countably many states $\mathfrak{C}(\mathbf{A},\overline{e})$. 
    It is then natural to ask the following questions:

    \bigskip

    \begin{itemize}
      \item For which choices of $\mathbf{A}$, $\overline{e}$ can we find
        a machine $\mathcal{A}$ as a subautomaton of 
        $\mathfrak{C}(\mathbf{A}, \overline{e})$?

      \item For which $\overline{e}$ is 
        $\mathcal{G}(\mathcal{A}) \cong \mathfrak{C}(\mathbf{A},\overline{e})$
        as an automaton?

      \item When is $\mathcal{G}(\mathcal{A}_1)$ a subgroup of 
        $\mathcal{G}(\mathcal{A}_2)$?

      \item If $\mathcal{A}$ is a subautomaton of 
        $\mathfrak{C}(\mathbf{A}, \overline{e})$, then where is it located?
        That is, for a state $\alpha \in \mathcal{A}$, what is 
        $\varphi(\alpha) \in \Z^m$?

      \item Is there a way to remove this parameter $\overline{e}$?
    \end{itemize}

    \bigskip

    It turns out that all of these questions are deeply connected, and
    understanding the answer requires understanding these lattices not
    at the level of abelian groups, but at the level of Modules.
  }

  \block{The Key}
  {
    One (perhaps surprising) fact that prompted this investigation is the
    presence of a machine, called the \textbf{Principal Automaton} 
    $\mathfrak{A}(\mathbf{A})$ living inside every 
    $\mathfrak{C}(\mathbf{A},\overline{e})$. Indeed, we present the following
    facts (which are partial answers to the above questions):

    \bigskip

    \begin{itemize}
      \item $\mathfrak{A}(\mathbf{A})$ can be found in 
        $\mathfrak{C}(\mathbf{A}, \overline{e})$ for every $\overline{e}$

      \item $\mathcal{G}(\mathfrak{A})$ is a subgroup of 
        $\mathcal{G}(\mathcal{A})$ whenever $\mathcal{A}$ and $\mathfrak{A}$
        share a matrix.

      \item There is a state $\delta \in \mathfrak{A}$ such that
        $\varphi(\delta) = \overline{e}$ in 
        $\mathfrak{C}(\mathbf{A}, \overline{e})$
    \end{itemize}
  }

  \block{The Technique}
  {
    While $\mathbf{A} : 2 \Z \oplus \Z^{m-1} \to \Z^m$ needs an even vector
    as input to ensure an integer vector output, $\mathbf{A}^{-1}$ can take
    in any integer and output an integer vector. With this in mind, we can
    (almost) make $\mathfrak{C}(\mathbf{A},\overline{e})$ into a
    $\mathbb{Z}[x]$-module, where the action $x \cdot \overline{v}$ is 
    given by $\mathbf{A}^{-1} \overline{v}$. In actuality, the action
    is only defined for polynomials with odd constant term.

    \bigskip

    Since $\chi_\mathbf{A}$ is irreducible, we see that this module is cyclic,
    indeed it is generated by $\overline{e}_1$, the first standard basis vector. 
    We define for a polynomial (with odd coefficient) $p$

    \[ p \cdot \mathcal{G} = \mathfrak{C}(\mathbf{A}, p \cdot \overline{e}_1) \]

    Then if $pq = r$, we have 
    $p \cdot \mathcal{G} \hookrightarrow r \cdot \mathcal{G}$ by the
    canonical inclusion $\overline{v} \mapsto q \cdot \overline{v}$, which
    preserves both the group structure and the residuation structure.
  }

  \block{The Answers}
  {
    By using a cycle in $\mathcal{A}$, we can write an equation $p = q$
    in two variables $\overline{v}$ and $\overline{e}$. Thus we can solve
    for $\overline{e}$ in terms of $\overline{v}$, answering some of our 
    questions.
  }

%%}}}

%% The right side {{{
  \column{0.25}
  \block{An Important Example: $\mathcal{A}^3_2$}
  {
    \begin{center}
    \begin{minipage}[c]{7in}
    \begin{tikzpicture}[scale=0.5]
    \tikzstyle{every node}+=[inner sep=0pt]
    \draw [black] (26.6,-12.1) circle (3);
    \draw (26.6,-12.1) node {$f$};
    \draw [black] (17.4,-25.6) circle (3);
    \draw (17.4,-25.6) node {$f_1$};
    \draw [black] (35.7,-25) circle (3);
    \draw (35.7,-25) node {$f_0$};
    \draw [black] (16.648,-22.708) arc (-174.31762:-254.2298:9.658);
    \fill [black] (16.65,-22.71) -- (17.07,-21.86) -- (16.07,-21.96);
    \draw (17.67,-14.97) node [left] {$1/0$};
    \draw [black] (33.761,-27.277) arc (-48.14861:-128.09563:11.117);
    \fill [black] (33.76,-27.28) -- (32.83,-27.44) -- (33.5,-28.18);
    \draw [black] (29.501,-12.822) arc (67.8011:2.5992:10.451);
    \fill [black] (29.5,-12.82) -- (30.05,-13.59) -- (30.43,-12.66);
    \draw [black] (32.758,-24.474) arc (-108.88288:-180.71682:9.832);
    \fill [black] (32.76,-24.47) -- (32.16,-23.74) -- (31.84,-24.69);
    \draw (27.31,-22.21) node [left] {$0/1$};
    \end{tikzpicture}
    \end{minipage}%
    \begin{minipage}[c]{\textwidth-7in}
      $\mathbf{A} = \begin{pmatrix} -1 & 1 \\ -\frac{1}{2} & 0 \end{pmatrix}$

      \bigskip
      \bigskip

      $\del_0 f = f_0$

      $\del_1 f = f_1$

      $\del_0 f_0 = \del_1 f_0 = f$

      $\del_0 f_1 = \del_1 f_1 = f_0$

    \end{minipage}
    \end{center}

    \begin{center}
    (Unlabeled edges correspond to both $0/0$ and $1/1$ edges)
    \end{center}
  }

  \block{Computing $f(0110\ldots)$}
  {
    \begin{center}
      \begin{minipage}[c]{7in}
        \begin{tikzpicture}[scale=0.25]
        \tikzstyle{every node}+=[inner sep=0pt]
        \draw [black, fill=blue!30] (26.6,-12.1) circle (3);
        \draw (26.6,-12.1) node {$f$};
        \draw [black] (17.4,-25.6) circle (3);
        \draw (17.4,-25.6) node {$f_1$};
        \draw [black] (35.7,-25) circle (3);
        \draw (35.7,-25) node {$f_0$};
        \draw [black] (16.648,-22.708) arc (-174.31762:-254.2298:9.658);
        \fill [black] (16.65,-22.71) -- (17.07,-21.86) -- (16.07,-21.96);
        \draw (17.67,-14.97) node [left] {$1/0$};
        \draw [black] (33.761,-27.277) arc (-48.14861:-128.09563:11.117);
        \fill [black] (33.76,-27.28) -- (32.83,-27.44) -- (33.5,-28.18);
        \draw [black] (29.501,-12.822) arc (67.8011:2.5992:10.451);
        \fill [black] (29.5,-12.82) -- (30.05,-13.59) -- (30.43,-12.66);
        \draw [black] (32.758,-24.474) arc (-108.88288:-180.71682:9.832);
        \fill [black] (32.76,-24.47) -- (32.16,-23.74) -- (31.84,-24.69);
          \draw (27.31,-22.21) node [left] {$\textcolor{red}{0}/1$};
        \end{tikzpicture}
      \end{minipage}%
      \begin{minipage}[c]{\textwidth-7in}
        $f(\textcolor{red}0110\ldots)$
      \end{minipage}
    \end{center}

    \bigskip
    \bigskip

    \begin{center}
      \begin{minipage}[c]{7in}
        \begin{tikzpicture}[scale=0.25]
        \tikzstyle{every node}+=[inner sep=0pt]
        \draw [black] (26.6,-12.1) circle (3);
        \draw (26.6,-12.1) node {$f$};
        \draw [black] (17.4,-25.6) circle (3);
        \draw (17.4,-25.6) node {$f_1$};
        \draw [black, fill=blue!30] (35.7,-25) circle (3);
        \draw (35.7,-25) node {$f_0$};
        \draw [black] (16.648,-22.708) arc (-174.31762:-254.2298:9.658);
        \fill [black] (16.65,-22.71) -- (17.07,-21.86) -- (16.07,-21.96);
        \draw (17.67,-14.97) node [left] {$1/0$};
        \draw [black] (33.761,-27.277) arc (-48.14861:-128.09563:11.117);
        \fill [black] (33.76,-27.28) -- (32.83,-27.44) -- (33.5,-28.18);
        \draw [black] (29.501,-12.822) arc (67.8011:2.5992:10.451);
        \fill [black] (29.5,-12.82) -- (30.05,-13.59) -- (30.43,-12.66);
        \draw [black] (32.758,-24.474) arc (-108.88288:-180.71682:9.832);
        \fill [black] (32.76,-24.47) -- (32.16,-23.74) -- (31.84,-24.69);
          \draw (27.31,-22.21) node [left] {$0/\textcolor{red}{1}$};
        \end{tikzpicture}
      \end{minipage}%
      \begin{minipage}[c]{\textwidth-7in}
        \textcolor{red}{1}$f_0(110\ldots)$
      \end{minipage}
    \end{center}

    \bigskip
    \bigskip

    \begin{center}
      \begin{minipage}[c]{7in}
        \begin{tikzpicture}[scale=0.25]
        \tikzstyle{every node}+=[inner sep=0pt]
        \draw [black] (26.6,-12.1) circle (3);
        \draw (26.6,-12.1) node {$f$};
        \draw [black] (17.4,-25.6) circle (3);
        \draw (17.4,-25.6) node {$f_1$};
        \draw [black, fill=blue!30] (35.7,-25) circle (3);
        \draw (35.7,-25) node {$f_0$};
        \draw [black] (16.648,-22.708) arc (-174.31762:-254.2298:9.658);
        \fill [black] (16.65,-22.71) -- (17.07,-21.86) -- (16.07,-21.96);
        \draw (17.67,-14.97) node [left] {$1/0$};
        \draw [black] (33.761,-27.277) arc (-48.14861:-128.09563:11.117);
        \fill [black] (33.76,-27.28) -- (32.83,-27.44) -- (33.5,-28.18);
        \draw [black] (29.501,-12.822) arc (67.8011:2.5992:10.451);
        \fill [black] (29.5,-12.82) -- (30.05,-13.59) -- (30.43,-12.66);
        \draw [black] (32.758,-24.474) arc (-108.88288:-180.71682:9.832);
        \fill [black] (32.76,-24.47) -- (32.16,-23.74) -- (31.84,-24.69);
        \draw (27.31,-22.21) node [left] {$0/1$};
        \end{tikzpicture}
      \end{minipage}%
      \begin{minipage}[c]{\textwidth-7in}
        $1 f_0(110\ldots)$
      \end{minipage}
    \end{center}

    \bigskip
    \bigskip

    \begin{center}
      \begin{minipage}[c]{7in}
        \begin{tikzpicture}[scale=0.25]
        \tikzstyle{every node}+=[inner sep=0pt]
        \draw [black, fill=blue!30] (26.6,-12.1) circle (3);
        \draw (26.6,-12.1) node {$f$};
        \draw [black] (17.4,-25.6) circle (3);
        \draw (17.4,-25.6) node {$f_1$};
        \draw [black] (35.7,-25) circle (3);
        \draw (35.7,-25) node {$f_0$};
        \draw [black] (16.648,-22.708) arc (-174.31762:-254.2298:9.658);
        \fill [black] (16.65,-22.71) -- (17.07,-21.86) -- (16.07,-21.96);
        \draw (17.67,-14.97) node [left] {$1/0$};
        \draw [black] (33.761,-27.277) arc (-48.14861:-128.09563:11.117);
        \fill [black] (33.76,-27.28) -- (32.83,-27.44) -- (33.5,-28.18);
        \draw [black] (29.501,-12.822) arc (67.8011:2.5992:10.451);
        \fill [black] (29.5,-12.82) -- (30.05,-13.59) -- (30.43,-12.66);
        \draw [black] (32.758,-24.474) arc (-108.88288:-180.71682:9.832);
        \fill [black] (32.76,-24.47) -- (32.16,-23.74) -- (31.84,-24.69);
        \draw (27.31,-22.21) node [left] {$0/1$};
        \end{tikzpicture}
      \end{minipage}%
      \begin{minipage}[c]{\textwidth-7in}
        $11 f(10\ldots)$
      \end{minipage}
    \end{center}

    \bigskip
    \bigskip

    \begin{center}
      \begin{minipage}[c]{7in}
        \begin{tikzpicture}[scale=0.25]
        \tikzstyle{every node}+=[inner sep=0pt]
        \draw [black] (26.6,-12.1) circle (3);
        \draw (26.6,-12.1) node {$f$};
        \draw [black, fill=blue!30] (17.4,-25.6) circle (3);
        \draw (17.4,-25.6) node {$f_1$};
        \draw [black] (35.7,-25) circle (3);
        \draw (35.7,-25) node {$f_0$};
        \draw [black] (16.648,-22.708) arc (-174.31762:-254.2298:9.658);
        \fill [black] (16.65,-22.71) -- (17.07,-21.86) -- (16.07,-21.96);
        \draw (17.67,-14.97) node [left] {$1/0$};
        \draw [black] (33.761,-27.277) arc (-48.14861:-128.09563:11.117);
        \fill [black] (33.76,-27.28) -- (32.83,-27.44) -- (33.5,-28.18);
        \draw [black] (29.501,-12.822) arc (67.8011:2.5992:10.451);
        \fill [black] (29.5,-12.82) -- (30.05,-13.59) -- (30.43,-12.66);
        \draw [black] (32.758,-24.474) arc (-108.88288:-180.71682:9.832);
        \fill [black] (32.76,-24.47) -- (32.16,-23.74) -- (31.84,-24.69);
        \draw (27.31,-22.21) node [left] {$0/1$};
        \end{tikzpicture}
      \end{minipage}%
      \begin{minipage}[c]{\textwidth-7in}
        $110 f_1(0\ldots)$
      \end{minipage}
    \end{center}
  }

  \block{An Embedding}
  {
      For $\overline{e} = (-3,-2)$:

      \bigskip

      $f = (1,0)$ 

      $f_0 = (0,1)$ 

      $f_1 = (-2,-2)$
  }
%%}}}
\end{columns}


\end{document}
