\documentclass[24pt]{tikzposter}

\geometry{paperwidth=48in, paperheight=36in}
\makeatletter
\setlength{\TP@visibletextwidth}{\textwidth-2\TP@innermargin}
\setlength{\TP@visibletextheight}{\textheight-2\TP@innermargin}
\makeatother

%{{{ Preamble

%%%%%%%%{{{ Packages
%%%%%%%%%%%%%%%%%%%%

% we aren't animals - we might use more than ASCII
\usepackage[utf8]{inputenc}
\usepackage[T1]{fontenc}

% obligatory math environments, symbols, and theorems
\usepackage{amsmath, amssymb, amsthm}

% moar symbols
\usepackage{stmaryrd}

% obligatory citation library
\usepackage{natbib}

% sometimes you gotta draw stuff, like _c_ommutative _d_iagrams.
\usepackage{tikz, tikz-cd}

% sometimes you gotta put in pretty pictures
\usepackage{graphicx}

% sometimes you gotta write code
\usepackage{listings}

% proof trees are useful
\usepackage{proof}

% use \mathbbm for bb-numerals, use \bm for bold math symbols.
\usepackage{bbm, bm}

% I like clickable links in pdfs
\usepackage{hyperref}

% Convenient todo-notes and missing figure boxes
\usepackage{todonotes}

% remove paragraph indentation
\usepackage{parskip}

%}}}

%%%%%%%%{{{ Formatting
%%%%%%%%%%%%%%%%%%%%%%

% prevent orphans/widows
\clubpenalty = 10000
\widowpenalty = 10000

% never break words across lines -- hyphens are stupid
\hyphenpenalty 10000
\exhyphenpenalty 10000

%}}}

%%%%%%%%{{{ Environments
%%%%%%%%%%%%%%%%%%%%%%%%

\newtheorem{thm}{Theorem}

\theoremstyle{definition}
\newtheorem{defn}{Definition}

%}}}

%%{{{ Aliases and Commands
%%%%%%%%%%%%%%%%%%%%%%%%%%

%{{{ blackboard letters
\newcommand{\N}{\mathbb{N}}
\newcommand{\Z}{\mathbb{Z}}
\newcommand{\Q}{\mathbb{Q}}
\newcommand{\R}{\mathbb{R}}
\newcommand{\C}{\mathbb{C}}
%}}}

%{{{ categories
\newcommand*{\catFont}[1]{\mathsf{#1}} 
\newcommand*{\catVarFont}[1]{\mathcal{#1}}

\newcommand{\Set}{\catFont{Set}}
\newcommand{\Top}{\catFont{Top}}
\newcommand{\Cat}{\catFont{Cat}}
\newcommand{\Grp}{\catFont{Grp}}
\newcommand{\Mod}{\catFont{Mod}}
\newcommand{\Sub}{\catFont{Sub}}
\newcommand{\FP}{\catFont{FP}}
\newcommand{\Pos}{\catFont{Pos}}
\newcommand{\FinSet}{\catFont{FinSet}}

\newcommand{\cata}{\catVarFont{A}}
\newcommand{\catb}{\catVarFont{B}}
\newcommand{\catc}{\catVarFont{C}}
\newcommand{\catd}{\catVarFont{D}}
\newcommand{\catx}{\catVarFont{X}}
\newcommand{\caty}{\catVarFont{Y}}
\newcommand{\catZ}{\catVarFont{Z}}
%}}}

%{{{ "operators" (words in math mode)
\DeclareMathOperator{\Hom}{Hom}
\DeclareMathOperator{\End}{End}
\DeclareMathOperator{\Aut}{Aut}
\DeclareMathOperator{\im}{im}
\DeclareMathOperator{\coker}{coker}
%}}}

%{{{ arrows
\newcommand{\hookr}{\hookrightarrow}
\newcommand{\hookl}{\hookleftarrow}
\newcommand{\monor}{\rightarrowtail}
\newcommand{\monol}{\leftarrowtail}
\newcommand{\epir}{\twoheadrightarrow}
\newcommand{\epil}{\twoheadleftarrow}
\newcommand{\regepir}{\rightarrowtriangle}
\newcommand{\regepil}{\leftarrowtriangle}
%}}}

%{{{ categorical limits
\newcommand{\limr}{\varinjlim}
\newcommand{\liml}{\varprojlim}
%}}}

%{{{ analysis
\DeclareMathOperator{\dif}{d \!}
\DeclareMathOperator{\Dif}{D \!}
\newcommand{\del}{\partial}
\newcommand*{\abs}[1]{\left \lvert #1 \right \rvert}
\newcommand*{\norm}[1]{\left \lVert #1 \right \rVert}
\newcommand*{\eval}[1]{\left . #1 \right \rvert}

% partial derivative command (taken from commath package)
% usage: \pd[n]{f}{x}
\newcommand*{\pd}[3][]{\ensuremath{
\ifinner
\tfrac{\partial{^{#1}}#2}{ \partial{#3^{#1}} }
\else
\dfrac{\partial{^{#1}}#2}{ \partial{#3^{#1}} }
\fi
}}

% ordinary derivative command (taken from commath package)
% usage: \od[n]{f}{x}
\newcommand*{\od}[3][]{\ensuremath{
\ifinner
\tfrac{\dif{^{#1}}#2}{ \dif{#3^{#1}} }
\else
\dfrac{\dif{^{#1}}#2}{ \dif{#3^{#1}} }
\fi
}}

%}}}

%{{{ algebra
\newcommand{\meet}{\wedge}
\newcommand{\join}{\vee}
\newcommand{\id}{\mathrm{id}}
%}}}

%{{{ topology
\newcommand*{\interior}[1]{ {\kern0pt#1}^{\mathrm{o}} }
%}}}

%{{{ logic
\renewcommand{\diamond}{\lozenge}
\newcommand*{\denote}[1]{\llbracket #1 \rrbracket}
%}}}

%{{{ misc symbols
\newcommand{\teq}{\triangleq}
\newcommand{\fin}{ \subseteq_{\text{fin}} }
\newcommand{\lleft}{\llbracket}
\newcommand{\rright}{\rrbracket}

% define a "danger" symbol for use when something surprising might occur
% https://tex.stackexchange.com/questions/159669/
% 	how-to-print-a-warning-sign-triangle-with-exclamation-point
% use outside of math mode!

\newcommand*{\TakeFourierOrnament}[1]{}}

%}}}

%% project specific aliases (if they exist)
\IfFileExists{../preamble.tex}{\input{../preamble.tex}}{}

%}}}

%% Heading
\author{Chris Grossack}
\title{Extensions of Abelian Automata Groups}
\author{Chris Grossack (advised by Klaus Sutner)}
\institute{Carnegie Mellon University}

\begin{document}
\maketitle

\begin{columns}
%% The left side {{{
  \column{0.25}
  \block{Mealy Automata}
  {
    A \textbf{Mealy Automaton} $\mathcal{A}$ is a finite state 
    machine which encodes a family of continuous functions from 
    Cantor Space to itself. For us, these continuous functions 
    will always be homeomorphisms, and thus we may associate
    to a machine $\mathcal{A}$ a subgroup $\mathcal{G}(\mathcal{A})$
    of the automorphisms of Cantor Space.

    \bigskip

    Classifying all groups generated by even 3-state machines is still
    and open problem, so we will focus attention on those which generate
    abelain groups.
  }

  \block{An Important Example: $\mathcal{A}^3_2$}
  {
    \begin{center}
    \begin{tikzpicture}[scale=0.5]
    \tikzstyle{every node}+=[inner sep=0pt]
    \draw [black] (26.6,-12.1) circle (3);
    \draw (26.6,-12.1) node {$f$};
    \draw [black] (17.4,-25.6) circle (3);
    \draw (17.4,-25.6) node {$f_1$};
    \draw [black] (35.7,-25) circle (3);
    \draw (35.7,-25) node {$f_0$};
    \draw [black] (16.648,-22.708) arc (-174.31762:-254.2298:9.658);
    \fill [black] (16.65,-22.71) -- (17.07,-21.86) -- (16.07,-21.96);
    \draw (17.67,-14.97) node [left] {$1/0$};
    \draw [black] (33.761,-27.277) arc (-48.14861:-128.09563:11.117);
    \fill [black] (33.76,-27.28) -- (32.83,-27.44) -- (33.5,-28.18);
    \draw [black] (29.501,-12.822) arc (67.8011:2.5992:10.451);
    \fill [black] (29.5,-12.82) -- (30.05,-13.59) -- (30.43,-12.66);
    \draw [black] (32.758,-24.474) arc (-108.88288:-180.71682:9.832);
    \fill [black] (32.76,-24.47) -- (32.16,-23.74) -- (31.84,-24.69);
    \draw (27.31,-22.21) node [left] {$0/1$};
    \end{tikzpicture}
    \end{center}

    \begin{center}
    (Unlabeled edges correspond to both $0/0$ and $1/1$ edges)
    \end{center}
  }

  \block{Computing $f(0110\ldots)$}
  {
    \begin{center}
      \begin{minipage}[c]{7in}
        \begin{tikzpicture}[scale=0.25]
        \tikzstyle{every node}+=[inner sep=0pt]
        \draw [black, fill=blue!30] (26.6,-12.1) circle (3);
        \draw (26.6,-12.1) node {$f$};
        \draw [black] (17.4,-25.6) circle (3);
        \draw (17.4,-25.6) node {$f_1$};
        \draw [black] (35.7,-25) circle (3);
        \draw (35.7,-25) node {$f_0$};
        \draw [black] (16.648,-22.708) arc (-174.31762:-254.2298:9.658);
        \fill [black] (16.65,-22.71) -- (17.07,-21.86) -- (16.07,-21.96);
        \draw (17.67,-14.97) node [left] {$1/0$};
        \draw [black] (33.761,-27.277) arc (-48.14861:-128.09563:11.117);
        \fill [black] (33.76,-27.28) -- (32.83,-27.44) -- (33.5,-28.18);
        \draw [black] (29.501,-12.822) arc (67.8011:2.5992:10.451);
        \fill [black] (29.5,-12.82) -- (30.05,-13.59) -- (30.43,-12.66);
        \draw [black] (32.758,-24.474) arc (-108.88288:-180.71682:9.832);
        \fill [black] (32.76,-24.47) -- (32.16,-23.74) -- (31.84,-24.69);
          \draw (27.31,-22.21) node [left] {$\textcolor{red}{0}/1$};
        \end{tikzpicture}
      \end{minipage}%
      \begin{minipage}[c]{\textwidth-7in}
        $f(\textcolor{red}0110\ldots)$
      \end{minipage}
    \end{center}

    \bigskip
    \bigskip

    \begin{center}
      \begin{minipage}[c]{7in}
        \begin{tikzpicture}[scale=0.25]
        \tikzstyle{every node}+=[inner sep=0pt]
        \draw [black] (26.6,-12.1) circle (3);
        \draw (26.6,-12.1) node {$f$};
        \draw [black] (17.4,-25.6) circle (3);
        \draw (17.4,-25.6) node {$f_1$};
        \draw [black, fill=blue!30] (35.7,-25) circle (3);
        \draw (35.7,-25) node {$f_0$};
        \draw [black] (16.648,-22.708) arc (-174.31762:-254.2298:9.658);
        \fill [black] (16.65,-22.71) -- (17.07,-21.86) -- (16.07,-21.96);
        \draw (17.67,-14.97) node [left] {$1/0$};
        \draw [black] (33.761,-27.277) arc (-48.14861:-128.09563:11.117);
        \fill [black] (33.76,-27.28) -- (32.83,-27.44) -- (33.5,-28.18);
        \draw [black] (29.501,-12.822) arc (67.8011:2.5992:10.451);
        \fill [black] (29.5,-12.82) -- (30.05,-13.59) -- (30.43,-12.66);
        \draw [black] (32.758,-24.474) arc (-108.88288:-180.71682:9.832);
        \fill [black] (32.76,-24.47) -- (32.16,-23.74) -- (31.84,-24.69);
          \draw (27.31,-22.21) node [left] {$0/\textcolor{red}{1}$};
        \end{tikzpicture}
      \end{minipage}%
      \begin{minipage}[c]{\textwidth-7in}
        \textcolor{red}{1}$f_0(110\ldots)$
      \end{minipage}
    \end{center}

    \bigskip
    \bigskip

    \begin{center}
      \begin{minipage}[c]{7in}
        \begin{tikzpicture}[scale=0.25]
        \tikzstyle{every node}+=[inner sep=0pt]
        \draw [black] (26.6,-12.1) circle (3);
        \draw (26.6,-12.1) node {$f$};
        \draw [black] (17.4,-25.6) circle (3);
        \draw (17.4,-25.6) node {$f_1$};
        \draw [black, fill=blue!30] (35.7,-25) circle (3);
        \draw (35.7,-25) node {$f_0$};
        \draw [black] (16.648,-22.708) arc (-174.31762:-254.2298:9.658);
        \fill [black] (16.65,-22.71) -- (17.07,-21.86) -- (16.07,-21.96);
        \draw (17.67,-14.97) node [left] {$1/0$};
        \draw [black] (33.761,-27.277) arc (-48.14861:-128.09563:11.117);
        \fill [black] (33.76,-27.28) -- (32.83,-27.44) -- (33.5,-28.18);
        \draw [black] (29.501,-12.822) arc (67.8011:2.5992:10.451);
        \fill [black] (29.5,-12.82) -- (30.05,-13.59) -- (30.43,-12.66);
        \draw [black] (32.758,-24.474) arc (-108.88288:-180.71682:9.832);
        \fill [black] (32.76,-24.47) -- (32.16,-23.74) -- (31.84,-24.69);
        \draw (27.31,-22.21) node [left] {$0/1$};
        \end{tikzpicture}
      \end{minipage}%
      \begin{minipage}[c]{\textwidth-7in}
        $1 f_0(110\ldots)$
      \end{minipage}
    \end{center}

    \bigskip
    \bigskip

    \begin{center}
      \begin{minipage}[c]{7in}
        \begin{tikzpicture}[scale=0.25]
        \tikzstyle{every node}+=[inner sep=0pt]
        \draw [black, fill=blue!30] (26.6,-12.1) circle (3);
        \draw (26.6,-12.1) node {$f$};
        \draw [black] (17.4,-25.6) circle (3);
        \draw (17.4,-25.6) node {$f_1$};
        \draw [black] (35.7,-25) circle (3);
        \draw (35.7,-25) node {$f_0$};
        \draw [black] (16.648,-22.708) arc (-174.31762:-254.2298:9.658);
        \fill [black] (16.65,-22.71) -- (17.07,-21.86) -- (16.07,-21.96);
        \draw (17.67,-14.97) node [left] {$1/0$};
        \draw [black] (33.761,-27.277) arc (-48.14861:-128.09563:11.117);
        \fill [black] (33.76,-27.28) -- (32.83,-27.44) -- (33.5,-28.18);
        \draw [black] (29.501,-12.822) arc (67.8011:2.5992:10.451);
        \fill [black] (29.5,-12.82) -- (30.05,-13.59) -- (30.43,-12.66);
        \draw [black] (32.758,-24.474) arc (-108.88288:-180.71682:9.832);
        \fill [black] (32.76,-24.47) -- (32.16,-23.74) -- (31.84,-24.69);
        \draw (27.31,-22.21) node [left] {$0/1$};
        \end{tikzpicture}
      \end{minipage}%
      \begin{minipage}[c]{\textwidth-7in}
        $11 f(10\ldots)$
      \end{minipage}
    \end{center}

    \bigskip
    \bigskip

    \begin{center}
      \begin{minipage}[c]{7in}
        \begin{tikzpicture}[scale=0.25]
        \tikzstyle{every node}+=[inner sep=0pt]
        \draw [black] (26.6,-12.1) circle (3);
        \draw (26.6,-12.1) node {$f$};
        \draw [black, fill=blue!30] (17.4,-25.6) circle (3);
        \draw (17.4,-25.6) node {$f_1$};
        \draw [black] (35.7,-25) circle (3);
        \draw (35.7,-25) node {$f_0$};
        \draw [black] (16.648,-22.708) arc (-174.31762:-254.2298:9.658);
        \fill [black] (16.65,-22.71) -- (17.07,-21.86) -- (16.07,-21.96);
        \draw (17.67,-14.97) node [left] {$1/0$};
        \draw [black] (33.761,-27.277) arc (-48.14861:-128.09563:11.117);
        \fill [black] (33.76,-27.28) -- (32.83,-27.44) -- (33.5,-28.18);
        \draw [black] (29.501,-12.822) arc (67.8011:2.5992:10.451);
        \fill [black] (29.5,-12.82) -- (30.05,-13.59) -- (30.43,-12.66);
        \draw [black] (32.758,-24.474) arc (-108.88288:-180.71682:9.832);
        \fill [black] (32.76,-24.47) -- (32.16,-23.74) -- (31.84,-24.69);
        \draw (27.31,-22.21) node [left] {$0/1$};
        \end{tikzpicture}
      \end{minipage}%
      \begin{minipage}[c]{\textwidth-7in}
        $110 f_1(0\ldots)$
      \end{minipage}
    \end{center}
  }

%%}}}

%% The center {{{
  \column{0.5}
  \block{The Center}
  {lorem ipsum}
%%}}}

%% The right side {{{
  \column{0.25}
  \block{The Right Side}
  {lorem ipsum}
%%}}}
\end{columns}


\end{document}
