\documentclass[12]{article}

\usepackage{amsmath,amsthm,amssymb}

\newcommand{\A}{\mathcal{A}}
\newcommand{\G}{\mathcal{G}}
\newcommand{\P}{\mathfrak{A}}
\newcommand{\C}{\mathfrak{C}}
\newcommand{\Z}{\mathbb{Z}}
\newcommand{\Q}{\mathbb{Q}}
\newcommand{\2}{\textbf{2}}
\newcommand{\Am}\textbf{A}}
\newcommand{\del}{\partial}
\newcommand{\v}{\bar{v}}

\newtheorem{thm}{Theorem}

\title{Extensions of Abelian Automata Groups}
\author{Chris Grossack}

\begin{document}
\maketitle

\section{Background}
TODO: this whole section

\section{Z[x] Module}
Going forward, $\G$ will denote $\G(\P)$ for some principal machine $\P$.

In the case of interest, $\G \cong \Z^m$. Further, $\del_0$ extends to 
a $\frac{1}{2}$-Integral matrix $\Am$ of irreducible character. Then
$\G$ admits representation as a $\Z[x]$ module where 
$x \cdot \v = \Am^{-1}\v$, extended linearly.
Further, since $\Am$ has irreducible character (in $\Z$ and in $\Q$), 
so does $\Am^{-1}$. Thus this module is cyclic, 
and generated by $\bar{e_1} = \delta$.

Clearly $End_{\G}$, the ring of module endomorphisms of $\G$, 
is isomorphic to $\Z[x]/(\chi^*)$, 
where $\chi^*$ is the characteristic polynomial of $\Am^{-1}$.
Now for some $p \in End_{\G}$ we consider
$p \cdot \G = \Z^m$, with $\delta$ at $p \cdot \bar{e_1}$.
This corresponds to using a residuation vector $\bar{e} = p \cdot bar{e_1}$.
(Thus we require $p$ have odd constant term)
For reasons which will soon become clear, we call $p \cdot \G$ the
\textbf{Group Extension} of $\G$ by $p$.

To justify this nomenclature, we first notice 
$\G \hookrightarrow p \cdot \G$ for all $p$ by the
homomorphism $\v \mapsto p \cdot \v$. 
Further, we recognize that if $p$ is not a unit in $End_{\G}$, 
this homomorphism is \emph{not} surjective. 
That is to say $\G$ is a proper subgroup of $p \cdot \G$.
\textbf{Note:} It \emph{is} the case that $\G \cong \Z^m \cong p \cdot \G$. 

Every function $f \in \G$ is also an element of $p \cdot \G$. 
If $f \in \G$ is located at $\v$, then $\f \in p \cdot \G$ is located
at $p \cdot \v$. In fact, this is true in a more general sense:

\begin{thm}
  If $mp = q$, then $p \cdot \G \hookrightarrow q \cdot \G$, 
  with a canonical injection $\phi_m~:~\v \mapsto m \cdot \v$. 
  In particular, if $m$ is a unit, then $p \cdot \G \cong q \cdot \G$.
\end{thm}

\begin{proof}
  Let $mp = q$, $f \in p \cdot \G$ located at $\v$.
  Consider $f' \in q \cdot \G$ located at $m \cdot \v$.

  First note $f$ and $f'$ have the same parity, since 
  $m$ has odd constant term, and so $\v$ and $m \cdot \v$
  have the same parity. Now, consider the residuals of $f$ and $f'$. 
  
  If $f$ is even, then 
  \[ \del_0 f' = \Am (m \cdot \v) = m \cdot \Am \v = m \cdot \del_0 f \]

  If $f$ is odd, then
  \[ \del_0 f' = \Am (m \cdot \v - q \cdot \bar{e_1}) 
               = m \cdot \Am (\v - p \cdot \bar{e_1})
               = m \cdot \del_0 f \]

  A similar argument shows $\del_1 f' = m \cdot \del_1 f$
\end{proof}

\end{document}
