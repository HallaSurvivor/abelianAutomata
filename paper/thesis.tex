\documentclass[12pt]{article}

\usepackage{amsmath,amsthm,amssymb}

\newcommand{\A}{\mathcal{A}}
\newcommand{\G}{\mathcal{G}}
\renewcommand{\P}{\mathfrak{A}}
\newcommand{\C}{\mathfrak{C}}
\newcommand{\Z}{\mathbb{Z}}
\newcommand{\Q}{\mathbb{Q}}
\newcommand{\2}{\textbf{2}}
\newcommand{\Am}{\textbf{A}}
\newcommand{\del}{\partial}
\renewcommand{\v}{\bar{v}}
\newcommand{\T}{\mathcal{T}}

\newtheorem{thm}{Theorem}

\title{Extensions of Abelian Automata Groups}
\author{Chris Grossack}

\begin{document}
\maketitle

\section{Background}
TODO: this whole section

\section{Group Extensions}
Going forward, $\G$ will denote $\G(\P)$ for some principal machine $\P$.

In the case of interest, $\G \cong \Z^m$. Further, $\del_0$ extends to 
a $\frac{1}{2}$-Integral matrix $\Am$ of irreducible character. Then
$\G$ admits representation as a $\Z[x]$ module where 
$x \cdot \v = \Am^{-1}\v$, extended linearly.
Further, since $\Am$ has irreducible character (in $\Z$ and in $\Q$), 
so does $\Am^{-1}$. Thus this module is cyclic, 
and generated by $\bar{e_1} = \delta$.

Clearly $\text{End}_{\G}$, the ring of module endomorphisms of $\G$, 
is isomorphic to $\Z[x]/(\chi^*)$, 
where $\chi^*$ is the characteristic polynomial of $\Am^{-1}$.
Now for some $p \in End_{\G}$ we consider
$p \cdot \G = \Z^m$, equipped with
$\del_0 \v = \Am (\v - p \cdot \bar{e_1})$, and 
$\del_1 \v = \Am (\v + p \cdot \bar{e_1})$
(thus we require $p$ have odd constant term).

(TODO: cases for above)

For reasons which will soon become clear, we call $p \cdot \G$ the
\textbf{Group Extension} of $\G$ by $p$.

To justify this nomenclature, we first notice 
$\G \hookrightarrow p \cdot \G$ for all $p$ by the
homomorphism $\v \mapsto p \cdot \v$. 
Further, we recognize that if $p$ is not a unit in $End_{\G}$, 
this homomorphism is \emph{not} surjective. 
That is to say $\G$ is a proper subgroup of $p \cdot \G$.
\textbf{Note:} It \emph{is} the case that $\G \cong \Z^m \cong p \cdot \G$. 

Every function $f \in \G$ is also an element of $p \cdot \G$. 
If $f \in \G$ is located at $\v$, then $f \in p \cdot \G$ is located
at $p \cdot \v$. In fact, this is true in a more general sense:

\begin{thm}
  If $mp = q$, then $p \cdot \G \hookrightarrow q \cdot \G$, 
  with a canonical injection $\phi_m~:~\v \mapsto m \cdot \v$. 
  In particular, if $m$ is a unit, then $p \cdot \G \cong q \cdot \G$.
\end{thm}

\begin{proof}
  Let $mp = q$, $f \in p \cdot \G$ located at $\v$.
  Consider $f' \in q \cdot \G$ located at $m \cdot \v$.

  First note $f$ and $f'$ have the same parity, since 
  $m$ has odd constant term, and so $\v$ and $m \cdot \v$
  have the same parity. Now, consider the residuals of $f$ and $f'$. 
  
  If $f$ is even, then 
  \[ \del_0 f' = \Am (m \cdot \v) = m \cdot \Am \v = m \cdot \del_0 f \]

  If $f$ is odd, then
  \[ \del_0 f' = \Am (m \cdot \v - q \cdot \bar{e_1}) 
               = m \cdot \Am (\v - p \cdot \bar{e_1})
               = m \cdot \del_0 f \]

  A similar argument shows $\del_1 f' = m \cdot \del_1 f$
\end{proof}

\section{Fractional Elements}
So for $p \cdot \G$ we know exactly how vectors of the form $p \cdot \v$
behave as functions: They are exactly the functions present in $\G$.
However there are plenty of vectors which do not have the above form. 
How do they behave? We call such vectors (and their corresponding functions)
\textbf{Fractional}, due to the following observation.

Consider $3 \cdot \G$. What does $\bar{e_1}$ do as a function?
Well, $3\bar{e_1}$ should do the same function as $\delta$, and so
$\bar{e_1}$ should behave like ``$\frac{1}{3}\delta$'', and in fact it does.

In general, $\v \in p \cdot \G$ behaves like $p^{-1} \cdot \v \in \G$,
(where $p^{-1}$ comes from $\Q[x]$ and so $p^{-1} \cdot \v \in \Q^m$)
and so Group Extensions give us access to fractional functions from 
our base group $\G$.

\begin{thm}
  $\forall p \in Z[x]$ with odd constant term,\\
  $\{ p^{-1} \cdot \v~|~\v \in \Z^m \} (\leq \Q^m) \cong p \cdot \G$.\\
  Further, this isomorphism respects $\del_0$ and $\del_1$.
\end{thm}

\begin{proof}
  Let $p^{-1} \cdot \Z^m = \{ p^{-1} \cdot \v~|~\v \in \Z^m \}$\\
  Consider $\varphi~:~p^{-1} \cdot \Z^m \to p \cdot \G$ by
  $\varphi(p^{-1} \cdot \v) = \v$.\\
  $\varphi$ is clearly bijective, and is a homomorphism since:
  \begin{align*}
       \varphi(p^{-1} \cdot \v_1 + p^{-1} \cdot \v_2) 
    &= \varphi(p^{-1} \cdot (\v_1 + \v_2))\\
    &= \v_1 + \v_2\\
    &= \varphi(p^{-1} \cdot \v_1) + \varphi(p^{-1} \cdot \v_2) 
  \end{align*}

  Further, if $\v$ is even, then:

  \begin{align*}
       \varphi(\del_0 p^{-1} \cdot \v)
    &= \varphi(\Am p^{-1} \cdot \v)\\
    &= \varphi(p^{-1} \cdot \Am \v)\\
    &= \Am \v\\
    &= \del_0 \varphi(p^{-1} \cdot \v)
  \end{align*}

  If $\v$ is odd, then: 

  \begin{align*}
       \varphi(\del_0 p^{-1} \cdot \v)
    &= \varphi(\Am (p^{-1} \cdot \v - \bar{e_1}))\\
    &= \varphi(p^{-1} \cdot \Am (\v - p \cdot \bar{e_1}))\\
    &= \Am (\v - p \cdot \bar{e_1})\\
    &= \del_0 \varphi(p^{-1} \cdot \v)
  \end{align*}

  The $\del_1$ proof is similar, and has been omitted.
\end{proof}

Thus we can view functions in $p \cdot \G$ as fractions of functions in $\G$.
It is a natural question to ask which fractions are attainable in this way.

Clearly, for any $f \in \G$, we can attain $\frac{1}{k} f$ for any odd $k$.
However, even fractions are unattainable, even in limiting cases. 
This is clear, since $\frac{1}{2}\delta$, for instance, must 
when applied twice, flip the first bit of the input. But it must itself
be even or odd. In either case, $2 \frac{1}{2} \delta$ does not flip the 
first bit, a contradiction.


\section{Characterizing Automata}
Every strongly connected automaton $\A$ lives in $p \cdot \G$ where 
$\G$ is the principal group associated to $\A$ and $p$ is a polynomial. 
Further, $\A$ can always be anchored with an odd state at $\bar{e_1}$. 
Thus, the study of Group Extensions is sufficient to study all abelian automata.

\begin{thm}
  Every strongly connected abelian automaton $\A$ can be anchored with an 
  odd state at $\bar{e_1}$ in $p \cdot \G$ as above.
\end{thm}

\begin{proof}
  Let $f$ be an odd state in $\A$. Then $f_0$ and $f_1$ are distinct
  and both have paths to $f$ by strongly connectedness. Thus we have two
  distinct cycles $c_0$ and $c_1$ from $f$ to $f$ in $\A$.
  Each of these cycles gives rise to a matrix equation 
  $\v_f = g_0(\v_f,\bar{e}) = g_1(\v_f,\bar{e})$ corresponding to following
  the path in $\Z^m$ (with unknown residuation vector $\bar{e}$). 
  Then $(g_0 - g_1)(\v_f, \bar{e}) = \bar{0}$ and we can solve for 
  $\bar{e}$ in terms of $\v_f$.

  Choosing $\v_f = \bar{e_1}$ gives a value for the residuation vector $e$,
  which (by cyclicity) gives a polynomial $p_e$ 
  such that $p_e \cdot \bar{e_1} = e$. Then, by construction, $\A$ is 
  a subautomaton of $p_e \cdot \G$, and is anchored with $f$ at $\bar{e_1}$.
  As desired.
\end{proof}

\section{Characterizing Orbits}

Recall an \textbf{Orbit} of $f \in \G(\A)$ at $u \in \2^{\omega}$
is $\{ f^t u | t \in \Z \}$, or, additively, $\{ tf u | t \in \Z \}$.
It is a reasonable question to wonder what these orbits look like
for an arbitrary automaton?

By the above theorem, it suffices to consider $f \in p \cdot \G$
for some principal group $\G$. Fix such a group.

We begin with a useful lemma

\begin{thm}
  $x^n \cdot \delta (u0v) = u1v$ (where $|u| = n$)
\end{thm}

\begin{proof}
  If $n=0$ the theorem is clear, since 
  $\delta 0v = 1 (\del_0 \delta v) = 1 (I v) = 1v$

  Further by induction, 
  $x^{n+1} \cdot \delta (u_0u0v) = 
  \Am^{-1} (x^n \cdot \delta) (u_0u0v) =
  u_0 (x^n \cdot \delta) (u0v) =
  u_0u1v$
\end{proof}

We will first prove the result in the finite case.

\begin{thm}
  For every word $u \in \2^n$, there exists a unique function
  (mod $Stab(0^n)$) $f$ such that $f 0^n = u$.
\end{thm}

\begin{proof}
  The existance of such a function is a direct consequence of the lemma,
  and is given by $\sum_{i=0}^{n-1} u_i x^i \delta$.
  Uniqueness mod $Stab(0^n)$ is immediate from basic group theory.
\end{proof}

Denote this function by $\langle u \rangle$. 

\begin{thm}
  The orbit of $f$ at $u$ is given by the line $\Z f + \langle u \rangle$
\end{thm}

\begin{proof}
  Let $w = tf u$. Then $\langle w \rangle$ sends $0^n$ to $w$, 
  but so does $tf + \langle u \rangle$. By uniqueness, then,
  $\langle w \rangle = tf + \langle u \rangle$ and the theorem follows.
\end{proof}

Note that this argument works in the infinite case as well, provided 
we have a suitable definition of $\langle u \rangle$ for $u \in \2^\omega$.
For cardinality reasons, we clearly cannot send $0^\omega$ to any $u$.
Below we will characterize exactly those $u$ for which a $\langle u \rangle$
exists.

\begin{thm}
  For $u \in \2^\omega$, $\langle u \rangle$ exists in some group extension
  iff $u$ is ultimately periodic.
\end{thm}

\begin{proof}
  Since every function $f \in p \cdot G$ eventualy residuates into a 
  strongly connected component, $f 0^\omega$ is ultimately periodic for
  every such $f$.

  Further, if we have an ultimately periodic string $u = tv^*$, then
  $\sum_{i=0}^{\infty} u_i x^i \delta$ works if it exists. Further,
  it exists, since:

  \begin{align*}
    \sum_{i=0}^{\infty} u_i x^i \delta 
    &= \sum_{i=0}^{|t|} t_i x^i \delta 
        + \sum_{i=|t|}^{\infty} v^*_i x^i \delta\\
    &= \sum_{i=0}^{|t|} t_i x^i \delta 
        + x^{|t|} \sum_{i=0}^{\infty} v^*_i x^i \delta\\
    &= \sum_{i=0}^{|t|} t_i x^i \delta 
         + x^{|t|} \frac{\sum_{i=0}^{|v|} v_i x^i \delta}{1 - x^{|v|}}\\
    &= \left ( 
        \sum_{i=0}^{|t|} t_i x^i
        + x^{|t|} \frac{\sum_{i=0}^{|v|} v_i x^i}{1 - x^{|v|}}
       \right ) \cdot \delta\\
    &= \frac%
        {%
          1 - x^{|v|} \sum_{i=0}^{|t|} t_i x^i + 
          x^{|t|} \sum_{i=0}^{|v|} v_i x^i
        }
        {1 - x^{|v|}}
       \cdot \delta\\
    &= \frac{q}{1 - x^{|v|}} \cdot \delta
  \end{align*}

  Finally, this sum is convergent in the cantor topology, since 
  for each $N \geq 0$, for all $n \geq N$, the $nth$ partial sums 
  agree on the first $N$ bits of the string.

  Thus, this sum is equal to $q \cdot \delta \in 1 - x^{|v|} \cdot \G$.
\end{proof}

For ultimately periodic strings $u \in \2^\omega$, $f$ orbits
of $u$ correspond to lines $\Z f + \langle u \rangle$.

\section{Graph Theoretic Insight}
Equipped with this module-theoretic background, 
we have a solid understanding of the algebraic structure of 
abelian automata groups. However, this ignores the combinatorial
aspect of these groups, and there are several interesting open 
problems regarding graph theoretic properties of these groups.

One problem which is surprisingly open is to bound the size of 
$|\P|$, the principal machine, without explicitly constructing it.
Okano showed that this is related to the spectral radius of $\Am$,
however a tight bound has been hard to find. Before we can describe
our bound, however, we must make a diversion.

\subsection{Tiling}
TODO: write more background

$F = \{\lambda \v . \Am (\v - \bar{e_1})
     ,\lambda \v . \Am (\v + \bar{e_1})
     ,\lambda \v . \Am \v
     \}$

Iterated Function Systems, Attractor, etc.

\subsection{A bound on $|\P|$}

\begin{thm}
  $\P \subseteq \Z^m \cap \T$, where $\T$ is the attractor of the IFS given by $F$.
\end{thm}

\begin{proof}
  Note $\P \subseteq F(\P)$, since functions in $F$ corresopnd to residuation.
  So certainly $\P \subseteq \T$, and $\T$ gives us a bound on the vectors
  in $\P$.
\end{proof}

\end{document}
