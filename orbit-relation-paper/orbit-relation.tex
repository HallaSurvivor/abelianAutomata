\documentclass[12pt]{article}

%{{{ Preamble

%%%%%%%%{{{ Packages
%%%%%%%%%%%%%%%%%%%%

% the default margins have always felt big to me
\usepackage[margin=1in]{geometry}

% we aren't animals - we might use more than ASCII
\usepackage[utf8]{inputenc}
\usepackage[T1]{fontenc}

% obligatory math environments, symbols, and theorems
\usepackage{amsmath, amssymb, amsthm}

% moar symbols
\usepackage{stmaryrd}

% obligatory citation library
\usepackage{natbib}

% sometimes you gotta draw stuff, like _c_ommutative _d_iagrams.
\usepackage{tikz, tikz-cd}

% sometimes you gotta put in pretty pictures
\usepackage{graphicx}

% sometimes you gotta write code
\usepackage{listings}

% proof trees are useful
\usepackage{proof}

% use \mathbbm for bb-numerals, use \bm for bold math symbols.
\usepackage{bbm, bm}

% I like clickable links in pdfs
\usepackage{hyperref}

% Convenient todo-notes and missing figure boxes
\usepackage{todonotes}

% remove paragraph indentation
\usepackage{parskip}

% allow fancy stuff
\usepackage{fancyhdr}

%}}}

%%%%%%%%{{{ Formatting
%%%%%%%%%%%%%%%%%%%%%%

% prevent orphans/widows
\clubpenalty = 10000
\widowpenalty = 10000

% never break words across lines -- hyphens are stupid
\hyphenpenalty 10000
\exhyphenpenalty 10000

%}}}

%%%%%%%%{{{ Environments
%%%%%%%%%%%%%%%%%%%%%%%%

\newtheorem{thm}{Theorem}

\theoremstyle{definition}
\newtheorem{defn}{Definition}

%}}}

%%{{{ Aliases and Commands
%%%%%%%%%%%%%%%%%%%%%%%%%%

%{{{ blackboard letters
\newcommand{\N}{\mathbb{N}}
\newcommand{\Z}{\mathbb{Z}}
\newcommand{\Q}{\mathbb{Q}}
\newcommand{\R}{\mathbb{R}}
\newcommand{\C}{\mathbb{C}}
%}}}

%{{{ categories
\newcommand*{\catFont}[1]{\mathsf{#1}} 
\newcommand*{\catVarFont}[1]{\mathcal{#1}}

\newcommand{\Set}{\catFont{Set}}
\newcommand{\Top}{\catFont{Top}}
\newcommand{\Cat}{\catFont{Cat}}
\newcommand{\Grp}{\catFont{Grp}}
\newcommand{\Mod}{\catFont{Mod}}
\newcommand{\Sub}{\catFont{Sub}}
\newcommand{\FP}{\catFont{FP}}
\newcommand{\Pos}{\catFont{Pos}}
\newcommand{\FinSet}{\catFont{FinSet}}

\newcommand{\cata}{\catVarFont{A}}
\newcommand{\catb}{\catVarFont{B}}
\newcommand{\catc}{\catVarFont{C}}
\newcommand{\catd}{\catVarFont{D}}
\newcommand{\catx}{\catVarFont{X}}
\newcommand{\caty}{\catVarFont{Y}}
\newcommand{\catZ}{\catVarFont{Z}}
%}}}

%{{{ "operators" (words in math mode)
\DeclareMathOperator{\Hom}{Hom}
\DeclareMathOperator{\End}{End}
\DeclareMathOperator{\Aut}{Aut}
\DeclareMathOperator{\im}{im}
\DeclareMathOperator{\coker}{coker}
%}}}

%{{{ arrows
\newcommand{\hookr}{\hookrightarrow}
\newcommand{\hookl}{\hookleftarrow}
\newcommand{\monor}{\rightarrowtail}
\newcommand{\monol}{\leftarrowtail}
\newcommand{\epir}{\twoheadrightarrow}
\newcommand{\epil}{\twoheadleftarrow}
\newcommand{\regepir}{\rightarrowtriangle}
\newcommand{\regepil}{\leftarrowtriangle}
%}}}

%{{{ categorical limits
\newcommand{\limr}{\varinjlim}
\newcommand{\liml}{\varprojlim}
%}}}

%{{{ analysis
\DeclareMathOperator{\dif}{d \!}
\DeclareMathOperator{\Dif}{D \!}
\newcommand{\del}{\partial}
\newcommand*{\abs}[1]{\left \lvert #1 \right \rvert}
\newcommand*{\norm}[1]{\left \lVert #1 \right \rVert}
\newcommand*{\eval}[1]{\left . #1 \right \rvert}

% partial derivative command (taken from commath package)
% usage: \pd[n]{f}{x}
\newcommand*{\pd}[3][]{\ensuremath{
\ifinner
\tfrac{\partial{^{#1}}#2}{ \partial{#3^{#1}} }
\else
\dfrac{\partial{^{#1}}#2}{ \partial{#3^{#1}} }
\fi
}}

% ordinary derivative command (taken from commath package)
% usage: \od[n]{f}{x}
\newcommand*{\od}[3][]{\ensuremath{
\ifinner
\tfrac{\dif{^{#1}}#2}{ \dif{#3^{#1}} }
\else
\dfrac{\dif{^{#1}}#2}{ \dif{#3^{#1}} }
\fi
}}

%}}}

%{{{ algebra
\newcommand{\meet}{\wedge}
\newcommand{\join}{\vee}
\newcommand{\id}{\mathrm{id}}
%}}}

%{{{ topology
\newcommand*{\interior}[1]{ {\kern0pt#1}^{\mathrm{o}} }
%}}}

%{{{ logic
\renewcommand{\diamond}{\lozenge}
\newcommand*{\denote}[1]{\llbracket #1 \rrbracket}
\newcommand*{\code}[1]{\langle #1 \rangle}
%}}}

%{{{ misc symbols
\newcommand{\teq}{\triangleq}
\newcommand{\fin}{ \subseteq_{\text{fin}} }
\newcommand{\lleft}{\llbracket}
\newcommand{\rright}{\rrbracket}

% define a "danger" symbol for use when something surprising might occur
% https://tex.stackexchange.com/questions/159669/
% 	how-to-print-a-warning-sign-triangle-with-exclamation-point
% use outside of math mode!

\newcommand*{\TakeFourierOrnament}[1]{}}

%}}}

%% project specific aliases (if they exist)
\IfFileExists{../preamble.tex}{\input{../preamble.tex}}{}

\newcommand{\A}{\mathcal{A}}
\newcommand{\G}{\mathcal{G}}
\newcommand{\coord}[1]{\langle #1 \rangle}
\newcommand{\2}{\mathbf{2}}

%}}}

%% Heading
\author{Chris Grossack}
\title{On the Orbit Relation for the Natural Action of Abelian Automata Groups}

% \institute%
% {%
%   Carnegie Mellon University, Pittsburgh, USA\\
%   \email{cgrossac@alumni.cmu.edu}
% }

\begin{document}
\maketitle

\begin{abstract}
  \todo{write this. ``we solve this problem on a dense set of cantor space''}

  % \keywords{Abelian Automata \and Transducer \and Group Action \and Orbit}
\end{abstract}

\section{Background}

Automata Groups give a way of combinatorially encoding extremely rich
group theoretic structure. They take the form of subgroups of the group
of automorphisms of Cantor Space~\cite{NekrashevychSidki2004:AutomorphismsOfTheBinaryTree-HalfEndomorphisms}
and have proven a useful source of examples and counterexamples%
\cite{Nekrashevych2005:Self-SimilarGroups, Sidki2000:AutomorphismsOfOne-RootedTrees, GrigorchukNekrashevychSushchanski2000:AutomataDynamicalSystemsAndGroups}
in addition to their independent interest%
\cite{BondarenkoNekrashevychEtAl2009:GroupsGeneratedBy3StateAutomata, SutnerLewi2012:IteratingInvertibleBinaryTransducers, Sutner2018:AbelianAutomata, Grossack2019:FractionalElements}.
Indeed, automata groups provide examples of
finitely generated infinite torsion groups, with application to Burnside's
Problem~\cite{Gupta1983:OnTheBurnsideProblemForPeriodicGroups}, as well as
the only known examples of groups of intermediate growth~\cite{Grigorchuk2011:MilnorProblemOnGrowthOfGroups}.

It goes without saying that general automata groups can be extremely complicated%
\cite{BondarenkoNekrashevychEtAl2009:GroupsGeneratedBy3StateAutomata},
and this complexity is in part responsible for their utility in the discovery
of counterexamples. However, this complexity makes any attempt at 
characterizing \emph{all} automata groups extremely difficult. Because of this,
we will focus on the \emph{abelian} automata groups in this article, because
a theorem of Nekrashevych and Sidki%
\cite{NekrashevychSidki2004:AutomorphismsOfTheBinaryTree-HalfEndomorphisms}
allows us to convert problems about abelian automata groups into problems of
linear algebra. Before we move forward, however, let us recall some important 
definitions regarding automata groups.

For our purposes, a \textbf{Mealy Automaton} is a tuple 
$\A = (Q,\tau : Q \times \2 \to Q \times 2)$ where $Q$ is called the 
\textbf{State Set} and $\tau$ is called the \textbf{Transition Function}. 
Here $\2 = \{0,1\}$ is called the \textbf{Alphabet}. When $\A$ has finitely
many states (that is, when $\abs{Q}$ is finite), we call $\A$ a 
\textbf{Finite Automaton}. Additionally, the components of the output of $\tau$
define functions $\del_a : Q \to Q$ and $\underline{q} : \2 \to \2$:
We define $(\del_a q, \underline{q}(a)) = \tau(q,a)$, where $\del_a q$ is
called the $\mathbf{a}$\textbf{-residual} of $q$.

We depict automata by (labeled, directed, multi-)graphs where we have one
vertex for each state, and we add an edge labeled $a / b$ from $q_1$ to $q_2$
whenever $\tau(q_1,a) = (q_2,b)$. In the interest of removing clutter, if
$\del_0 q = \del_1 q$ and $\underline{q} = \id$, we write a single, unlabeled,
edge instead of two parallel edges.

As an example, the following depicts the automaton 
$\A^3_2 = (\{f,g,h\},\tau)$ where 
$\tau(f,0) = (g,1)$, $\tau(f,1) = (h,0)$, 
$\tau(g,a) = (h,a)$, $\tau(h,a) = (f,a)$:

\missingfigure{$\A^3_2$}

Notice the parallel arrows from $g$ to $h$ and from $h$ to $f$ have been 
replaced by one, unlabeled arrow, as per the convention above. Here we have
$\del_0 f = g$, $\del_1 f = h$, $\del_0 g = \del_1 g = h$, and 
$\underline{f}(0) = 1$, $\underline{f}(1) = 0$, 
$\underline{h}(0) = 0$, $\underline{h}(1) = 1$.

Finally, we extend $\underline{q}$ to a function on \emph{infinite} binary 
strings $\underline{q} : \2^\omega \to \2^\omega$ inductively by

\[ \underline{q}(ax) = \underline{q}(a) \underline{\del_a q}(x) \]

This function is best understood by putting one's finger on $q$ in the 
graphical representation of $\A$. When we read a bit $a$ from our input, we 
follow the edge labelled $a / b$, and we output the bit $b$. Now we are faced
with the rest of the (infinite) binary string, and our finger is on a new state
$\del_a q$. Bravely, we start the process again, and continue until the end of time.

\missingfigure{Steal the graphic from your other paper ``animating'' this}

These functions $\underline{q} : \2^\omega \to \2^\omega$ are easily seen to
be continuous, and they are known to be invertible whenever they are locally 
invertible~\cite{SutnerLewi2012:IteratingInvertibleBinaryTransducers}, 
so we restrict attention to those automata where $\underline{q} : \2 \to \2$ 
is invertible for every $q \in Q$. In this case, we can look at the subgroup 
$\G(\A) = \langle \underline{q} ~|~ q \in Q \rangle \leq \Aut(2^\omega)$
generated by these functions, which we call the \textbf{Automata Group} of
$\A$. 

Notice $\G(\A)$ can also be given the structure of an automaton. For
$f \in \G(\A)$, we may define 
$\tau(f,a) = (g,b)$ where $g$ is the unique function such that
$f(aw) = b g(w)$ for every $w \in \2^\omega$. This is well defined
since it is well defined on the generators, as is easily checked.
In this way, we see $\A$ is a subautomaton of $\G(\A)$, since 
$q \mapsto \underline{q}$ is an embedding which respects the transition
functions $\tau$. 

It is a theorem of Nekrashevych and Sidki%
\cite{NekrashevychSidki2004:AutomorphismsOfTheBinaryTree-HalfEndomorphisms}
  that the abelian automata groups are exactly $\Z^m$

\todo{Introduce notation $p^{-1} \cdot \G$ and assert N+S result on integer lattices}

\todo{Introduce the Orbit Problem}

\section{The Identifying Function}

\todo{come up wither a better name for $\coord{\overline{v}}$ than ``the identifying fn''?}
  
\todo{Introduce $\coord{\overline{v}}$}

\todo{Cite that old paper of Klaus?}

\todo{explain that the best we can do is ultimately periodic $\overline{v}$}

\todo{Prove it actually sends $0^\omega$ to $\overline{v}$}

\section{Solving the Orbit Problem}
  
\todo{Show two strings are in the same orbit iff $\coord{\overline{v}} - \coord{\overline{w}} = tf$}

\todo{Extend this result to a semidecidable procedure for all of $2^\omega$?}

\section{Conclusion}
  
\section*{Acknowledgements}

\todo{Say how much you love Klaus}

\newpage

\bibliography{\string~/.bib.bib}
\bibliographystyle{plain}

\end{document}
